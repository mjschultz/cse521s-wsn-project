% Matthew Lindsay, Patrick McBryde, Michael Schultz
% CSE521S Project Proposal

\documentclass[12pt]{article}
\usepackage{fancyhdr}
\pagestyle{fancy}
\usepackage[margin=1in]{geometry}

\lhead{Matthew Lindsay, Patrick McBryde, Michael Schultz}
\rhead{CSE 521S: Fall 2010}
\renewcommand{\headrulewidth}{0pt}
\fancyheadoffset{0.5in}

\begin{document}
\begin{center}
{\bf \large CSE 521S Project Proposal\\
A Parking Guidance and Information System for TinyOS}
\end{center}

``Intelligent Transportation Systems'' are a mix of systems that extract
information from commonly trafficked areas and then communicates that
information to a central location.
A part of those systems includes ``Parking guidance and information'' (PGI)
systems, that keep track of vehicular information in parking garages, typically
only counting the number of spaces available in a designated area.
While these solutions exist in the wild (e.g. the Brentwood Metrolink station,
keeps track of parking spot availability), they can lack versatility, 
be cost prohibitive, and be difficult to install.

This project proposes to build parking guidance and information system on the
TinyOS platform.
We will take advantage of the multi-hop routing abilities and a heterogeneous
mix of sensors to keep track of parking spot availability and usage data,
possible using other sensing metrics as needed.
In the end, we plan to have developed an extensible system able to provide the
same information as existing systems with the addition of being able to
support different sensors by using an open platform.
TinyOS will allows us to extend our monitoring to other metrics such as
headlight, heat, or magnetic field detection with great ease.
Also, unlike most existing solutions, our system will not require expensive
installation and simplifies hardware and software expansion.

To achieve these goals, we intend to use the Tmote Sky/TelosB mote platform
which have an expansion connector that allows external sensors to be connected.
Here we'll need to develop our software to detect a change in voltage from the
ADC inputs or a signal from the digital inputs.
As long as our external interface doesn't change we should be able to swap in
and out external sensors without problem.
To reduce installation costs, we will use a multi-hop routing protocol (likely
the collection tree protocol) to communicate data from a sensor to the base
station.
This allows for a large number of sensors to be deployed with little effort or
physical infrastructure to be in place.
Finally, once the data is aggregated at a central location it must be displayed
to the end user in an easy to read/interpret interface.
This is important because the vehicle operator needs to have a quick, intuitive
knowledge of where to go, so they don't cause physical congestion.
We also wish to monitor the duration of parking space use, to allow a parking
attendant to be notified when a person has overused their space.

We would like 5 TelosB/Tmote Sky sensor (2 for sensor testing and 3 for network
development to build a non-trivial routing topology).
It would also be interesting to use a Mica family board with the MTS310CA
(magnetometer) sensor board, to test vehicle detection with a magnetometer,
though not strictly required as we can find other sensors to use.
Alternatively, if we can interface a magnetometer with the TelosB/Tmote Sky
mote board that could be used instead of a Mica family board.

The major components of this project are building and developing the sensing
devices, writing and testing the networking and communications
software to handle data delivery, and creating a friendly front end to present
and track the information as needed.
If we discover any of these components are significantly easier than others, it
is easy to combine forces to develop new/better methods for any of them.
We also have an interest in discovering more specific information of existing
systems and getting up-to-date in the area of intelligent transportation
systems.  The specific breakdown of high-level duties and responsibilities is as
follows:

\begin{itemize}
    \item Matthew: Handle sensing methods and help develop software to read get
    sensed data on to the mote.
    \item Patrick: Handle data communication on the network to get sensed data
    to the base station.
    \item Michael: Handle moving information from the base station to a display
    method.
    \item All: Handle integrating the various components of the system in a
    meaningful way.
\end{itemize}

\end{document}
