% CSE 521S Final Report
% A Parking Guidance and Information System for TinyOS
% By Matthew Lindsay, Patrick McBryde, Michael Schultz

\documentclass{acm_proc}
\usepackage{hyperref}
\usepackage{url}
\usepackage{graphicx}

\begin{document}

\title{A Parking Guidance and Information System for TinyOS}
\subtitle{CSE 521S Final Report}

\numberofauthors{3}
\author{
\alignauthor Matthew Lindsay\\
% 2nd. author
\alignauthor Patrick McBryde\\
% 3rd. author
\alignauthor Michael Schultz\\
\and
\affaddr{Washington University in Saint Louis}\\
}

\maketitle

\begin{abstract}
``Intelligent Transportation Systems'' (ITS) are a mix of systems that
gather various data metrics from transportation areas (parking lots,
streets, alleyways, etc.), aggregate this data together, and present
coherent information to the end-users of the system.
Orchestrating such a system presents several challenges at each level.
How do you gather the data and at what granularity?
Where does the data go once gathered and can it be useful to anyone?
If it can be useful, how can it be presented to end-users to provide
accurate and simple-to-interpret content?
This paper presents and explains our decisions in developing a ``Parking
Guidance and Information'' (PGI) system for TinyOS.
\end{abstract}

\section{Introduction}

Driving to new destinations always brings some level of stress and
uncertainty.
Where am I going, what do the roads and intersections look like, what side
of the road is the place on, where do I park?
These are all questions that dart through the brain when beginning to think
about going some place new.
Luckily, maps bring us the answer or at least a partial answer to some of
those questions.
In the past decade there has been huge growth of web-based mapping
technology to help answer these questions even more fully.
For example, Google Maps allows you to get directions from point A to point
B, and in the past 5 years has introduced Street
View~\cite{vincent:streetview} to their interface.
Street View allows a user to view the roads they will be travelling on from
an in-car perspective.
This removes much of the uncertainty when driving, leading to less
confusion and a better experience (and potentially fewer accidents).

However, there is still that can be done.
Even after you've arrived at your destination you must find where to park.
This can also be quite a hassle, as you are unfamiliar with the area and
during peak hours there may not be a parking space in near proximity.
To help in this matter, Parking Guidance and Information Systems (PGIS)
allow drivers to quickly evaluate where best to go to find parking
efficiently~\cite{sakai:pgi-toyota}.
Traditional PGI Systems simply count the number of cars that enter and exit
a designated area and display the number of spots available in that area to
the driver.
These systems are imprecise, costly, and can't be integrated with other
technologies.

This paper presents our experiences building a parking guidance and
information system using the TelosB/TinyOS platform.
By using motes, it opens the possibility of deploying a single sensor for
each available parking space allowing for much more detailed information
about a parking lot as a whole.
Moreover, with proper sheltering, these sensors can be mounted on the
surface of a parking lot instead of cutting into the lot and installing an
inductive loop to detect vehicle presence.
We also take advantage of the wireless multi-hop routing abilities of
TinyOS-based motes to avoid wiring and enable a heterogeneous mix of
sensors to keep track of parking spot availability and usage data.
Combining these aspects gives a system that can be more precise, lower
cost, and more easily integrated with future technologies.

The remainder of this paper is as follows.
Section~\ref{sec:goals} defines the goals of this project more fully.
Section~\ref{sec:design} presents and explains the high-level architecture
of our PGI System and details the hardware and software used during the
course of this project.
Section~\ref{sec:experiment} talks about how we convinced ourselves that
the system worked correctly and, if we have more resources, could scale up
as needed.
Sections~\ref{sec:related} and~\ref{sec:lessons} give an overview of
related works and lessons we learned during this project.
Finally, Section~\ref{sec:conclusions} concludes this paper with a
discussion of potential future work for this project.

\section{Goals}\label{sec:goals}

Since we believe other solutions to PGI Systems do not represent what
modern technology is capable of, our goals for this project are to build a
PGI System that can be easily deployed at a low-cost, aggregates the data
at a single location, and integrates with user-facing technology to provide
a rich experience.
In short, we plan to answer the following questions:
\begin{itemize}
	\item How do you gather parking data and at what granularity?
	\item Where does the data go once gathered and can it be useful to
	anyone?
	\item How can it be presented to end-users to provide accurate and
	simple-to-interpret content?
\end{itemize}

% from proposal
%To achieve these goals, we intend to use the Tmote Sky/TelosB mote platform
%which have an expansion connector that allows external sensors to be
%connected.
%To reduce installation costs, we will use a multi-hop routing protocol
%(likely the collection tree protocol) to communicate data from a sensor to
%the base station.
%This allows for a large number of sensors to be deployed with little effort
%or physical infrastructure to be in place.
%Finally, once the data is aggregated at a central location it must be
%displayed to the end user in an easy to read/interpret interface.
%This is important because the vehicle operator needs to have a quick,
%intuitive knowledge of where to go, so they don't cause physical
%congestion.
%We also wish to monitor the duration of parking space use, to allow a
%parking attendant to be notified when a person has overused their space.
% end from proposal

\section{Design}\label{sec:design}

\begin{figure}
    \begin{center}
		\includegraphics[width=\columnwidth]{figures/high-level}
	\end{center}
	\caption{High-level view of our PGI System. A base station at the lot
	sends data over the Internet to our aggregator, which end-users can
	query.}
	\label{fig:high-level}
\end{figure}

Figure~\ref{fig:high-level} shows a high-level view of the infrastructure
of our PGI System.
Our system begins with a parking lot (left side of
Figure~\ref{fig:high-level}), each space in this lot is equipped with a
sensor built on the TelosB/TinyOS platform.
Data is collected at a per-lot base station, which then sends the data over
the Internet to our backend.
When our backend receives the data it updates and logs the changes to the
data store and organizes the lot information.
Once the data is stored, clients are able to query the backend and get
up-to-date information about a specific lot and lots in the vicinity.

Building this system requires a fair amount of hardware and software
knowledge and is described below.

\subsection{Hardware}

In order to monitor the status of individual parking spaces in a lot or garage we need reliable, low cost hardware

The TelosB mote was chosen to monitor 

This should cover the physical hardware we used in building this.
Including (but not limited to):

- TelosB motes (listing any relevant specifications, why we used TelosB
 and not something else) \cite{xbow:telosb-datasheet}.

- External sensor we used, specifications for that, more detail than
 TelosB since the reader will be less familiar with it, why we used that.

At least one pictures of the telosb, the sensors, or the telosb with sensor
attached.

\subsection{Collection Software}

Yup we had software.

A paragraph or two about TinyOS and building whatever we used there
(interfaces, interaction with other sensors and base
station, CTP)~\cite{tep119:collection}.

A paragraph about the base station software and what it does.

% from proposal
The major components of this project are building and developing the
sensing devices, writing and testing the networking and communications
software to handle data delivery, and creating a friendly front end to
present and track the information as needed.
If we discover any of these components are significantly easier than
others, it is easy to combine forces to develop new/better methods for any
of them.
We also have an interest in discovering more specific information of
existing systems and getting up-to-date in the area of intelligent
transportation systems.
% end from proposal

\subsection{Aggregation Software}

Data leaving the base station of a parking lot is directed over then
Internet to our aggregation software hosted on Google's
AppEngine~\cite{google:appengine}.
For this project, we decided the AppEngine environment was an appropriate
choice for several reasons:
\begin{itemize}
	\item The service is free for light usage (testing and development).
	\item It provides a Python programming environment.
	\item Most importantly, it is designed to scale up with ease
	transparently with large datasets.
\end{itemize}

The nature of our project is to have one sensor per parking space in a
parking lot for every parking lot.
This implies that over time our backend must be able to accept, store, and
recall large amounts of current and historical data.
We could have sunk our resources into using a traditional server model for
the purposes of a prototype, however if the project were to start scaling
up great amounts we would start running into resource barriers.
Google's AppEngine is designed to scale with demand, this is largely due to
their use of a datastore built on top of Bigtable~\cite{google:bigtable}.
Though transparent to users of our system, this non-SQL based datastore
forced a few notable differences from a SQL based datastore.

First, while queries appear to be SQL, they are actually ``GQL,'' a
SQL-like language.
In general this does not bring any problems, but because of the
organization of the datastore imprecise queries are not acceptable.
You must know the data you are looking for.
You cannot simple SELECT one or two columns from the datastore, rather you
must get all the columns or a unique key identifier that matches the query.

Second, its harder to perform proximity searches with the datastore because
there is not concept of ``similar'' or ``like'' queries, you either know
the data or don't know the data.
However, with the help of the geomodel API~\cite{geomodel} we are able to
use geohashing to perform proximity searches on the datastore.
This is useful to locate nearby parking lots, allowing the end-user to
identify other potential parking areas.

With an understanding of the underlying technology, we are able to present
an easy to use API.

\subsubsection{A RESTful API}

REpresentational State Transfer (REST) is an abstract model for building
large-scale web services~\cite{pautasso:restful}.
The principles of a RESTful architecture are an identification of resources
through a URI, a ``pure'' HTTP interface, and self-contained hyperlinks.
In this section we describe how we achieved these principles.

Our URI scheme is straight-forward.
Each lot is given a unique identification consisting of alphanumerics and
the underscore (\_) character.
When attempting to access a specific lot, that identifier is part of the
URI (i.e.,
\texttt{http://<host>/lot/wustl\_millbrook/}
identifies the Milbrook lot on Washington University's campus).

Any request concerning this lot must go through its URI.
Requests use one of the HTTP verbs of \texttt{GET}, \texttt{PUT}, or
\texttt{POST}.
A \texttt{GET} request simply returns the information for the requested URI
in either the HTML or JSON format.
A \texttt{PUT} request attempts to store updated status information to the
lot.
This request contains the list of parking spaces to update, their
identification and status, as well as any associated meta-information that
should be put in the datastore.
Finally, a \texttt{POST} request handles the creation of new parking lots.

With the API in place, we are able to create rich applications with ease
for expansion and integration into other software.
As a demonstration, we build up a web-based frontend in the next section
that takes advantage of our API.

\subsubsection{Web Frontend}

Explanation of the frontend will go here.

\section{Experiment}\label{sec:experiment}

Paragraph about experiments that show our system works.

\section{Related Work}\label{sec:related}

We are aware of two systems that target per-spot granularity.
The first, Signal-Park, uses sonar sensors above each parking spot to
detect the presence of a vehicle~\cite{pgi:signal-park}.
Approximately three times per second, the sensor is queried to determine
vehicle presences and updates a central computer.
This system uses serial lines to connect each sensor with the central
computer, leading to more costly deploy.
It also appears to be limited to the parking lot that it is deployed in and
lacks the ability to aggregate the information for others to benefit.
It would certainly be possible to improve the connectivity of this system
to integrate with our back-end aggregation system to take advantage of
existing installations.

The other system, Streetline, is a startup in the San Francisco area
that similarly uses a wireless mesh network to link all the sensors
together~\cite{pgi:streetline}.
The Streetline system has recently (Summer 2010) begun to roll out sensors
and upgraded meters for select areas in the San Francisco
area~\cite{wired:streetline, sfpark}.
This system uses surface (and sub-surface) sensor mounts that contain a
wireless transmitter and a magnetometer for vehicle detection.
Status is sent to a central aggregation server where users can see the
status of street parking via their web browser, text message, or smart
phone; it also allows municipal workers to identify vehicles that have not
paid fully.
Presuming the Streetline trial is successful and the system expands, our
system and their system should be able to integrate with little effort and
could be seen as competitors.
We view this a positive, as it demonstrates that our project is a useful
idea with potential buyers available.

\section{Lessons Learned}\label{sec:lessons}

Anything we would have done differently if we were to pursue this project
in more detail again.

\section{Conclusions and Future Work}\label{sec:conclusions}

Conclusions from this project.

% \section{References}
\bibliographystyle{abbrv}
\bibliography{report}

\end{document}
